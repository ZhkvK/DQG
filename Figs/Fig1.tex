% w_lj_standalone.tex
\documentclass[tikz, border=1mm]{standalone}

% ========== Настройка шрифта (pdfLaTeX) ==========
% Рекомендуем: TeX Gyre Heros — свободная Helvetica-like гарнитура.
\usepackage[T1]{fontenc}
\usepackage[utf8]{inputenc}
%\usepackage{tgheros}   % <--- поменяй здесь на \usepackage[scaled]{helvet} если хочешь PS Helvetica
\usepackage{arev}
\usepackage{helvet}
%\usepackage{sansmath}  % даёт \sansmath для sans-математики
\renewcommand{\familydefault}{\sfdefault}
\usepackage{sfmath}
\usepackage{textcomp}
\usepackage{upgreek}

% Подключаем pgfplots/tikz и нужные библиотеки
\usepackage{pgfplots}
\pgfplotsset{compat=1.18} % современный режим совместимости
\usepackage{pgfplotstable}
\usetikzlibrary{calc,arrows.meta,shapes,patterns,decorations.pathreplacing,backgrounds}
% (добавь другие библиотеки, если они тебе нужны)

% Математика и шрифты
\usepackage{amsmath,amssymb}

% ====== Применяем sans ко всему документу (без побочных эффектов на главный файл,
% поскольку это standalone) ======
\sffamily
%\sfmath % теперь все формулы будут в sans внутри этого документа

% --- при желании: уменьшить/увеличить масштаб шрифтов в рисунке ---
%\renewcommand{\normalsize}{\fontsize{9}{11}\selectfont}
%\renewcommand{small}{\fontsize{7}{11}\selectfont}

%\renewcommand{\pi}{\uppi}
%\renewcommand{\omega}{\upomega}

\pgfplotsset{width=8.5cm,
compat=1.3,
% define the layers you need.
% (Don't forget to add `main' somewhere in that list!!)
layers/my layer set/.define layer set={
	background,
	main,
	foreground
}{
	% you could state styles here which should be moved to
	% corresponding layers, but that is not necessary here.
	% That is why we don't state anything here
},
% activate the newly created layer set
set layers=my layer set,
scaled y ticks=true,
scaled x ticks=false,}

% ========== Тело документа: вставляй сюда исходный tikzpicture ==========
\begin{document}
	
	% --- Ниже вставлен (твой) исходный tikzpicture: ---
	% (я почти ничего в нём не менял, только обёрнул в standalone preamble)
	\pgfplotstableread{Data/Fig1/data.txt}\datatable
	\begin{tikzpicture}
		
		\definecolor{darkgrey176}{RGB}{176,176,176}
		\definecolor{firebrick}{RGB}{178,34,34}
		\definecolor{lightgrey204}{RGB}{204,204,204}
		\definecolor{steelblue}{RGB}{0, 102, 204}
		\definecolor{orange}{RGB}{255, 153, 51}
		
		\begin{axis}[  % Общая подпись Y слева,
%			width=0.5\textwidth,
			%		legend cell align={left},
			%		legend style={fill opacity=0.8, draw opacity=1, text opacity=1, draw=lightgrey204, at={([yshift=10pt]0.5,1)},
				%			anchor=south,},
			tick align=inside,
			tick pos=left,
			x grid style={darkgrey176},
			xlabel={Wave vector, $qn^{-1/3}/\pi$},
			ylabel={Frequency, $\mathsf{\omega}/\omega_0$},
			xmin=0,
			xtick style={color=black},
			%		xtick={0, 20, 40, 60},
			y grid style={darkgrey176},
			ymin=0,
			ytick style={color=black},
			%		ytick={0, 1},
			ylabel shift={-5pt},
			enlargelimits=false,
			axis lines=box,
			clip=true,
			clip mode=individual,
			%		every axis y label/.style={
				%			at={(ticklabel* cs:1.0)},
				%			xshift=0pt,
				%			rotate=90 % уменьшает расстояние
				%		},
			]
			\addplot [mark=o, only marks, line width=0.3mm, mark size=2pt, each nth point={5}, orange, on layer=background]
			table[x=q, y=w_T]{\datatable};
			\addplot [mark=o, only marks, line width=0.3mm, mark size=2pt, each nth point={5}, lightgrey204, on layer=background]
			table[x=q, y=w_L]{\datatable}
			;
			%		\addplot [blue, dashed, forget plot, on layer=foreground]
			%		table[x=q, y=w]{../vkr_norm/images/texfigs/lj/1.0/w_lj_5.0_1.00/wfit_1.txt};
			\addplot [red, thick, line width=0.3mm, forget plot, on layer=foreground]
			table[x=q, y=w]{Data/Fig1/fit1.txt};
			\addplot [gray, dashed, line width=0.3mm, forget plot, on layer=foreground]
			table[x=q, y=w]{Data/Fig1/fit2.txt}
			node [yshift=8pt,black,sloped, pos=0.25, font=\small]{$\omega = cq$};
			\draw[thick, color=firebrick] (axis cs:0.3987835617604114,0) -- (axis cs:0.3987835617604114,3.5);
			\draw[thick, <->, color=firebrick] (axis cs:0,3) -- 
			(axis cs:0.3987835617604114,3) node[midway, above, font=\small] {$q_g$};
			
			\fill [
			gray,           % Цвет
			opacity=0.35,   % Прозрачность (0 = полностью прозрачный, 1 = непрозрачный)
			%draw=red,      % Граница (опционально)
			thick, on layer=background
			] 
			(axis cs:0,0) % Нижняя граница (x=1.5, y=0)
			rectangle 
			(axis cs:0.126,\pgfkeysvalueof{/pgfplots/ymax});
		\end{axis}
	\end{tikzpicture}
	
\end{document}
