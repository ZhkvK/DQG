\documentclass[aps,prl,reprint,groupedaddress]{revtex4-2}

% Standard packages
\usepackage[english]{babel}
\usepackage[utf8]{inputenc}
\usepackage[T1]{fontenc}
\usepackage{graphicx}
\graphicspath{{Figs/}{./}} % Path to figure files

% Math packages
\usepackage{amsmath,amsfonts,amssymb,amsthm,mathtools}
\usepackage[version=4]{mhchem}

% Table packages
\usepackage{booktabs}
\usepackage{multirow}

\usepackage{indentfirst}

% Hyperlink setup
\usepackage{hyperref}
\hypersetup{
	colorlinks=true, % Whether to color the text of links
	urlcolor=blue, % Color for \url and \href links
	linkcolor=blue, % Color for \nameref links
	citecolor=blue, % Color of reference citations
}

\begin{document}
	
	\title{Supplementary material: Dependence of the self-diffusion coefficient on the $q$-gap value in fluids}
	
	\author{Konstantin P. Zhukov}
	\author{Nikita P. Kryuchkov}
	\email[]{kruchkov_nkt@mail.ru}
	\author{Stanislav O. Yurchenko}
	\affiliation{Bauman Moscow State Technical University, 2nd Baumanskaya Street 5, 105005 Moscow, Russia}
	
	\date{\today}
	
	\maketitle
	\onecolumngrid 
	
	\section{Molecular dynamics simulations details}
	We considered systems of particles interacting via the Lennard-Jones (LJ) and Yukawa (Ykw) pair potentials:
	\begin{equation}
		\varphi_{\text{LJ}}(r) = 4\varepsilon\left[\left(\dfrac{\sigma}{r}\right)^{12} - \left(\dfrac{\sigma}{r}\right)^6\right],
		\label{DQGSM:Eq1}
	\end{equation}
	\begin{equation}
		\varphi_{\text{Ykw}}(r) = A\dfrac{e^{-k_D r}}{r},
		\label{DQGSM:Eq2}
	\end{equation}
	where $ \varepsilon $ and $ \sigma $ are the energy and length scale of the interaction,
	respectively; $A$ and $k_D$ are parameters which where chosen as $A = 1$, $k_D = \lambda_D^{-1} = 1$. Corresponding coupling and screening parameters are $\Gamma = A(ak_BT)^{-1}$ and $\kappa = a\lambda_D^{-1}$, respectively. Here $\lambda_D$ is the Debye length, $a = (3/4\pi n)^{1/3}$ is the Wigner-Seitz radius, where $n = N/V$ is the numerical density, $k_B$ is the Boltzmann constant and $T$ is the temperature.
	
	We have studied 3D and 2D fluids consisting of $N = 4000$ ($N = 4900$ in 2D case) particles in an NVT ensemble with a Nose–Hoover thermostat. The initial state of the system was a cubic lattice (a square in 2D case) with a size $L = 10$ ($L = 70$ in 2D case) with periodic boundary conditions with Maxwell distribution of velocities. The cutoff radius of interaction was chosen as $r_\text{cut} = 7.5n^{-1/d}$, where $n = N/V$ is the numerical density and $d$ is the space dimension. The mass $m$, length $\sigma$, and energy $\varepsilon$ were normalized to unity.
	
	The interatomic potential with machine learning (ML-IAP) for liquid aluminum \cite{10.1103/PhysRevResearch.5.033162} was also considered:
	\begin{equation}
		\varphi_{\text{ML-IAP}}(r) = \varphi_{\text{ref}}(r) + \sum_{i=1}^{N} E^{i}_{\text{SNAP}},
	\end{equation}
	where $E_\text{ref}$ denotes a reference potential and $E^i_{\text{SNAP}}$ is the total energy of atom $i$ relative to the atoms in its neighborhood, governed by the spectral neighbor analysis potential method (SNAP) \cite{10.1016/j.jcp.2014.12.018}. Ziegler-Biersack-Littmark (ZBL) potential \cite{10.1007/978-1-4615-8103-1_3} is used as a reference potential. The SNAP energy for each atom is calculated using a linear combination of the bispectrum components $\mathbf{B}^i$
	\begin{equation}
		E^{i}_{\text{SNAP}} = \boldsymbol{\beta} \times \mathbf{B}^i,
	\end{equation}
	where $\boldsymbol{\beta}$ are linear coefficients and $\mathbf{B}^i$ are descriptors of the local environment of atom $i$. Obtained using machine learning coefficients \cite{10.14278/RODARE.2368} $\boldsymbol{\beta}$ were used in this work.
	
	Simulations of liquid \ce{Al} were performed in a three-dimensional face-centered cubic box with dimensions of $Lx = Ly = Lz = 10a$, where $a$ is lattice constant, with mass density $\rho = 2.7 \text{g/cm}^3$. The remaining simulation details including number of particles, thermostating and boundary conditions follow previously established setup.
	
	Moreover, we considered liquid iron using Embedded atom model (EAM). The general form of this potential can be expressed as:
	%
	\begin{equation}
		E_i = \sum_{i} F(\rho_i) + \frac{1}{2} \sum_{i} \sum_{j \neq i} \varphi_{\text{p}}(r_{ij}),
	\end{equation}
	where $F_{i}(\rho_{i})$ denotes the embedding energy for atom $i$, a function of the local electron density $\rho_{i}$. This local density is computed as a superposition of the contributions from all neighboring atoms $j$:
	
	\begin{equation}
		\rho_{i} = \sum_{j \neq i} \rho(r_{ij}).
	\end{equation}
	
	In these expressions, $\varphi_{\text{p}}(r_{ij})$ represents the pairwise potential between atoms $i$ and $j$, separated by a distance $r_{ij}$. The specific functional forms for $F(\rho)$, $\rho(r)$, and $\varphi_{\text{p}}(r)$ employed in this work are those established in the corresponding paper \cite{10.1103/PhysRevLett.84.3638}.
	
	The mass density $\rho = 10g/cm^3$ was used in the simulations for liquid iron. The remaining simulation parameters correspond to those employed for liquid iron.
	
	Simulations of fluids of particles interacting via the Lennard-Jones and Yukawa potentials were performed using the numerical time step $\Delta t = 5\times 10^{-3}\sqrt{m\sigma^2/\varepsilon}$. For ML-IAP and EAM potentials in liquid aluminum and iron the time step $\Delta t = 5$ fs was chosen. All simulations where performed with the LAMMPS package \cite{10.1006/jcph.1995.1039} for $5\times10^4$ time steps, where the first 5000 were used for system relaxation.
	\begin{table}[h!]
		\centering
		\caption{Chosen states of considered systems.}
		\label{DQGSM:Table1}
		\begin{tabular}{@{}cccc@{}}
			\toprule
			Potential & Dimension & Density & Temperature \\ \midrule
			ML-IAP (\ce{Al}) & 3D & 2.7 g/cm$^3$ & 1560 --- 2560 K \\
			\midrule
			EAM (\ce{Fe}) & 3D & 3.6 g/cm$^3$ & 1812 --- 2392 K \\
			\midrule
			\multirow{11}{*}{LJ} & \multirow{8}{*}{3D} & 1.00 & 1.6 --- 7.0 \\
			& & 1.10 & 2.7 --- 8.0 \\
			& & 1.20 & 4.0 --- 9.0 \\
			& & 1.30 & 5.8 --- 15.6 \\
			& & 1.40 & 8.1 --- 17.9 \\
			& & 1.60 & 14.6 --- 21.8 \\
			& & 1.80 & 24.2 --- 29.8 \\
			& & 2.00 & 37.4 -- 44.8 \\
			\cmidrule(l){2-4}
			& \multirow{3}{*}{2D} & 0.80 & 0.60 --- 3.10 \\
			& & 0.85 & 1.00 --- 3.50 \\
			& & 0.90 & 2.00 --- 4.50 \\ \midrule
			\multirow{12}{*}{Yukawa} & \multicolumn{1}{l}{} & $\kappa$ & $\Gamma$ \\ \cmidrule(l){3-4}
			& \multirow{7}{*}{3D} & 1.00 & 217 --- 117 \\
			& & 1.25 & 250 --- 110\\
			& & 1.50 & 270 --- 86\\
			& & 2.00 & 440 ---  157\\
			& & 2.50 & 750 --- 305\\
			& & 3.00 & 1185 --- 665\\ \cmidrule(l){2-4}
			& \multirow{5}{*}{2D} & 1.00 & 177 --- 77 \\
			& & 1.50 & 250 --- 73\\
			& & 2.00 & 395 --- 112 \\
			\bottomrule
		\end{tabular}
	\end{table}
	
	\begin{figure*}[h!]
		\centering
		\includegraphics[width=0.6\textwidth]{SM-Fig1}
		\caption{Dispersion relations $\omega_{L,T}(q)$ for a 3D Lennard-Jones fluid at density $n = 1$ and temperature $T = 5$. Transverse ($\omega_T$) and longitudinal ($\omega_L$) modes are represented by orange and gray circles, respectively. The gray and red lines depict the theoretical asymptotic curves $\omega = cq$. The dark gray zone indicates the region where $qn^{-1/3} < 2\pi/L$ and finite-size effects are significant ($L$ being the system size).}
		\label{DQGSM:Fig1}
	\end{figure*}
	
	\begin{figure*}[h!]
		\centering
		\includegraphics[width=0.6\textwidth]{SM-Fig2}
		\caption{Dependence of the MSD on the time $t$ for the Lennard-Jones potential in three- and two-dimensional cases.}
		\label{DQGSM:Fig2}
	\end{figure*}
	
	\begin{figure*}[h!]
		\centering
		\includegraphics[width=0.9\textwidth]{SM-Fig3}
		\caption{The comparison between $D(q_g)$ relations in extended (blue) and original (red) 3D systems with Lennard-Jones ($n=1$) and Yukawa ($\kappa=1$) potentials.}
		\label{DQGSM:Fig3}
	\end{figure*}
	
	\begin{table*}[h!]
		\centering
		\caption{Minimum values of $q_g$ and corresponding slopes for studied systems}
		\label{DQGSM:Table2}
		\begin{tabular}{@{}ccccc@{}}
			\toprule
			Potential & Dimension & Density & \begin{tabular}[c]{@{}c@{}}Value of $q_g$ \\ near the MP, $q_{0}$\end{tabular} & $\left( \dfrac{\text{d}D}{\text{d}q_g}\right)_{q_0}$ \\ \midrule
			ML-IAP (\ce{Al}) & 3D & 2.7 g/cm$^3$ & 0.078 & 0.013 \\
			\midrule
			EAM (\ce{Fe}) & 3D & 3.6 g/cm$^3$ & 0.168 & 0.024 \\
			\midrule
			\multirow{11}{*}{LJ} & \multirow{8}{*}{3D} & 1.0 & 0.165 &0.0166 \\
			& & 1.1 & 0.159 &0.0142 \\
			& & 1.2 & 0.157 &0.0136 \\
			& & 1.3 & 0.151 &0.0124 \\
			& & 1.4 & 0.151 &0.0110 \\
			& & 1.6 & 0.169 &0.0079 \\
			& & 1.8 & 0.182 &0.0074 \\
			& & 2.0 & 0.170 &0.0062 \\ \cmidrule(l){2-5}
			& \multirow{3}{*}{2D} & 0.80 & 0.172 & \\
			& & 0.85 & 0.119 &\\
			& & 0.90 & 0.056 & \\ \midrule
			\multirow{12}{*}{Yukawa} & \multicolumn{1}{l}{} & $\kappa$ \\ \cmidrule(l){3-3}
			& \multirow{7}{*}{3D} & 1.00 & 0.115 & 0.015\\
			& & 1.25 & 0.105 & 0.027\\
			& & 1.50 & 0.123 & 0.042\\
			& & 2.00 & 0.106 & 0.052\\
			& & 2.50 & 0.130 & 0.058\\
			& & 3.00 & 0.111 & 0.098\\
			\cmidrule(l){2-5}
			& \multirow{5}{*}{2D} & 1.00 & 0.090 \\
			& & 1.50 & 0.097 \\
			& & 2.00 & 0.129 \\ \bottomrule
		\end{tabular}
	\end{table*}
	
	\begin{figure}[h!]
		\centering
		\includegraphics[width=0.5\linewidth]{SM-Fig4}
		\caption{Dependence of the self-diffusion coefficient $D$ on the $q$-gap width for 3D liquid aluminum at density $\rho = 2.7$ g/cm$^3$.}
		\label{DQGSM:Fig4}
	\end{figure}
	
	\bibliography{DQG-bibliography_DOI}
\end{document}