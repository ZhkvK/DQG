\documentclass[tikz, border=1mm]{standalone}

% ========== Настройка шрифта (pdfLaTeX) ==========
% Рекомендуем: TeX Gyre Heros — свободная Helvetica-like гарнитура.
\usepackage[T1]{fontenc}
\usepackage[utf8]{inputenc}
%\usepackage{tgheros}   % <--- поменяй здесь на \usepackage[scaled]{helvet} если хочешь PS Helvetica
\usepackage{arev}
\usepackage{helvet}
%\usepackage{sansmath}  % даёт \sansmath для sans-математики
\renewcommand{\familydefault}{\sfdefault}
\usepackage{sfmath}
\usepackage{textcomp}
\usepackage{upgreek}

% Подключаем pgfplots/tikz и нужные библиотеки
\usepackage{pgfplots}
\pgfplotsset{compat=1.18} % современный режим совместимости
\usepackage{pgfplotstable}
\usetikzlibrary{calc,arrows.meta,shapes,patterns,decorations.pathreplacing,backgrounds,spy}
\usepgfplotslibrary{groupplots}
% (добавь другие библиотеки, если они тебе нужны)

% Математика и шрифты
\usepackage{amsmath,amssymb}

% ====== Применяем sans ко всему документу (без побочных эффектов на главный файл,
% поскольку это standalone) ======
\sffamily
%\sfmath % теперь все формулы будут в sans внутри этого документа

% --- при желании: уменьшить/увеличить масштаб шрифтов в рисунке ---
%\renewcommand{\normalsize}{\fontsize{9}{11}\selectfont}
%\renewcommand{small}{\fontsize{7}{11}\selectfont}

%\renewcommand{\pi}{\uppi}
%\renewcommand{\omega}{\upomega}

\pgfplotsset{width=8.5cm,
	compat=1.3,
	% define the layers you need.
	% (Don't forget to add `main' somewhere in that list!!)
	layers/my layer set/.define layer set={
		background,
		main,
		foreground
	}{
		% you could state styles here which should be moved to
		% corresponding layers, but that is not necessary here.
		% That is why we don't state anything here
	},
	% activate the newly created layer set
	set layers=my layer set,
	scaled y ticks=true,
	scaled x ticks=false,}

	\begin{document}
		\begin{tikzpicture}
	
	\definecolor{darkgray176}{RGB}{176,176,176}
	\definecolor{blu}{RGB}{0, 127, 210}        % Синий (голубой) — круги
	\definecolor{gree}{RGB}{75, 150, 0}           % Зелёный — треугольники вверх
	\definecolor{DustyRose}{RGB}{197, 94, 94}            % Тёмно-зелёный — треугольники вниз
	\definecolor{orang}{RGB}{255, 127, 0}         % Оранжевый — ромбы
	\definecolor{re}{RGB}{210, 0, 0}         % Красный — звёздочки
	\definecolor{purp}{RGB}{127, 0, 127} 
	\definecolor{lightgrey204}{RGB}{204,204,204}
	
	\begin{axis}[
		height=6cm,
%		width=0.6\textwidth,
		legend cell align={left},
		legend style={fill opacity=0.8, draw opacity=1, draw=none,, text opacity=1, at={([yshift=0pt]0.1,0.73)},
			anchor=south,},
		tick align=inside,
%		tick pos=left,
		x grid style={darkgray176},
%		xmajorgrids,
		xlabel={$ q $-gap, $q_g$},
		xmin=0,
		xtick style={color=black},
		y grid style={darkgray176},
%		ymajorgrids,
		grid style=dashed,
		ylabel={Self-diffusion, $D\cdot10^{-3}$},
%		ylabel shift=-5pt,
		ytick scale label code/.code={},
		yticklabel style={
			/pgf/number format/fixed,
			/pgf/number format/precision=2
		},
		xticklabel style={
			/pgf/number format/fixed,
			/pgf/number format/precision=2
		},
		ymin=0,
		]
%		\addplot [semithick, steelblue31119180, mark=*, only marks, mark size=3]
%		table[x=q,y=D]{../vkr_norm/images/Dvsq_new/Al/1.0/2.7.txt};
%		\addlegendentry{Single mode analysis};
%		\addplot [semithick, steelblue31119180, forget plot]
%		table[x=x,y=y]{../vkr_norm/images/Dvsq_new/Al/1.0/fit2.7.txt};
		\addplot [semithick, blu, line width=0.5mm, mark=o, only marks, mark size=2, mark options={solid}]
		table[x=q,y=D]{Data/Fig5/Al.txt};
		\addlegendentry{Al};
		\addplot [semithick, blu, forget plot]
		table[x=x,y=y]{Data/Fig5/fitAl.txt};
		
		\addplot [semithick, re, line width=0.5mm, mark=o, only marks, mark size=2, mark options={solid}]
		table[x=q,y=D]{Data/Fig5/Fe.txt};
		\addlegendentry{Fe};
		\addplot [semithick, re, forget plot]
		table[x=x,y=y]{Data/Fig5/fitFe.txt};
	\end{axis}
%	\spy [black] on (0,0) in node at (1,1);
\end{tikzpicture}
	\end{document}