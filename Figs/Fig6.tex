\documentclass[tikz, border=2mm]{standalone}
\usepackage[T1]{fontenc}
\usepackage[utf8]{inputenc}
%\usepackage{tgheros}   % <--- поменяй здесь на \usepackage[scaled]{helvet} если хочешь PS Helvetica
\usepackage{arev}
\usepackage{helvet}
%\usepackage{sansmath}  % даёт \sansmath для sans-математики
\renewcommand{\familydefault}{\sfdefault}
\usepackage{sfmath}
\usepackage{textcomp}
\usepackage{upgreek}

% Подключаем pgfplots/tikz и нужные библиотеки
\usepackage{pgfplots}
\pgfplotsset{compat=1.18} % современный режим совместимости
\usepackage{pgfplotstable}
\usetikzlibrary{calc,arrows.meta,shapes,patterns,decorations.pathreplacing,backgrounds,spy}
\usepgfplotslibrary{groupplots}
% (добавь другие библиотеки, если они тебе нужны)

% Математика и шрифты
\usepackage{amsmath,amssymb}

% ====== Применяем sans ко всему документу (без побочных эффектов на главный файл,
% поскольку это standalone) ======
\sffamily
%\sfmath % теперь все формулы будут в sans внутри этого документа

% --- при желании: уменьшить/увеличить масштаб шрифтов в рисунке ---
%\renewcommand{\normalsize}{\fontsize{9}{11}\selectfont}
%\renewcommand{small}{\fontsize{7}{11}\selectfont}

%\renewcommand{\pi}{\uppi}
%\renewcommand{\omega}{\upomega}

\pgfplotsset{width=8.5cm,
	compat=1.3,
	% define the layers you need.
	% (Don't forget to add `main' somewhere in that list!!)
	layers/my layer set/.define layer set={
		background,
		main,
		foreground
	}{
		% you could state styles here which should be moved to
		% corresponding layers, but that is not necessary here.
		% That is why we don't state anything here
	},
	% activate the newly created layer set
	set layers=my layer set,
	scaled y ticks=true,
	scaled x ticks=false,}

\begin{document}
	\begin{tikzpicture}
		\begin{axis}[
			% Основные настройки
%			width=12cm,
%			height=10cm,
			xlabel={$\left.\partial D/\partial q_g\right|_{q_0}$},
			ylabel={$Dn^{1/3}/c$},
			xmin=0,
			ymin=0,
			xticklabel style={
				/pgf/number format/fixed,
				/pgf/number format/precision=2
			},
			% Настройки колормапа
%			colormap/viridis, % Выберите нужную цветовую схему (viridis, hot, jet, etc.)
%			colorbar,
%			colorbar style={ylabel={Slope, $k$}},
			% Настройки точек
%			scatter,
%			scatter src=explicit, % Использовать явные значения из данных
%			point meta min=0, % Минимальное значение для цвета (автоопределение можно убрать)
%			point meta max=0.06, % Максимальное значение для цвета
			% Сетка и прочее
%			grid=major,
%			grid style={dotted, gray!30},
			legend style={
				fill opacity=1, draw opacity=1, text opacity=1, draw=none, at={([yshift=10pt]0.5,1.05)},
				anchor=south,},
			legend columns=2,
			legend cell align={left},
			]
			
			% Загрузка данных из файла
			\addplot[red, line width=0.5mm, only marks,
			mark size=2pt,
			mark=o,] table [
			y index=0,
			x index=2,
			] {/home/konst/lj_3d/Dqk_lj.txt};
			\addlegendentry{LJ (3D)};
			\addplot[blue, line width=0.5mm, only marks,
			mark size=2pt,
			mark=o,] table [
			y index=0,
			x index=2,
			] {/home/konst/lj_3d/Dqk_yukawa.txt};
			\addlegendentry{Yukawa (3D)};
%			\addplot table [
%			y index=0,
%			x index=2,
%			] {/home/konst/lj_3d/Dqk_lj2d.txt};
%			\addlegendentry{LJ (2D)};
			\addplot[orange, line width=0.5mm, only marks,
			mark size=2pt,
			mark=o,] table [
			y index=0,
			x index=2,
			] {/home/konst/lj_3d/Dqk_yukawa2d.txt};
			\addlegendentry{Yukawa (2D)};
			\addplot[green, line width=0.5mm, only marks,
			mark size=2pt,
			mark=o,] table [
			y index=0,
			x index=2,
			] {/home/konst/lj_3d/Dqk_Al.txt};
			\addlegendentry{Al (3D)};
			\addplot[black, line width=0.5mm, only marks,
			mark size=2pt,
			mark=o,] table [
			y index=0,
			x index=2,
			] {/home/konst/lj_3d/Dqk_Fe.txt};
			\addlegendentry{Fe (3D)};
		\end{axis}
	\end{tikzpicture}
\end{document}